	Out of the 4 algorithms we tried (SVM, Naive Bayes, Decision Tree and Markov Random Fields), 2 of them were Generative algorithms and 2 were Discriminative. Our Markov Random Field additionally used the conditional probabilities based on time and location of the tweets to give a similarity measure to the prediction. The training set we originally had from the Kaggle competition did not have timestamps on the tweets. This proved to be a problem for us in implementing the MRF learner, as location similarity is not enough to give an accurate classification. For instance, two particular tweets could be from the exact same location, but 6 months apart in time, therefore having different weather. 

	The research we did in bringing this project to completion allowed us to read about multiple algorithms, or modifications to existing algorithms that would give us a very robust classifier. In particular, we read about various Probabilistic Graphical Models that made the use of a ``grid-like'' arrangement of the tweets, with edges between tweets representing a dependency. The Markov Random Field algorithm we implemented was one such Probabilistic Graph Model. 

	In these models, the basic idea was that the tweets had certain independent variables, and certain dependent variables. In our situation, the independent variables were the class-conditional word probabilities (inferred from the training set word counts) and the independent variables were the location and time-based weights. Another such algorithm was the Conditional Random Field, which made the use of a similar kind of pattern recognition for structured predictions based on ``similar'' tweets. 

	The problem with most of these algorithms ended up being that they made assumptions about a sequence of observations being related. However, in our case, we had tweets that might be within the same timeframe but not related due to location differences, or vice versa. We also began with a limited dataset that did not have this vital information; and even with the Cheng-Caverlee-Lee dataset, we did not have latitudes and longitudes to introduce a proper distance-based similarity, but instead we had city and state names. Keeping in mind the constraints on our datasets, the Markov Random Field was the most appropriate to implement, using the conditional-dependencies as a Probabilistic Graphical Model.
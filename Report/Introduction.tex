	There are approximately 200 million active users on Twitter and 400 million tweets are posted on a daily basis. With such a large volume of tweets there is huge potential for gathering information about public sentiment. We would like to know whether our machine learning tools could viably be used to leverage this data and predict public opinion. The specific domain of weather is particularly rich – weather is a common topic of conversation, and this is no different in the Twitter sphere. For example, people may use twitter to express their joy about a beautiful sunny day or to complain about the consistent rain. Therefore, we aim to build a classifier that can extract a tweet's sentiment, time frame, and weather condition by tailoring our off-the-shelf learning algorithms to this unique setting. Then we can examine our different classifier to determine which algorithms are most effective.

            To build our classifier we used three different algorithms: decision trees, support vector machines (SVM) and Naive Bayes. Each algorithm was modified to fit our problem and model selection was used to determine the best-input parameters. Our findings showed that each of the algorithms had its pros and cons with better performance in different aspects of the problem. In addition, we attempted to build a classifier using a Markov Random Field, which would be an improvement on the Naive Bayes algorithm that factors in a tweets location and time of posting. In this report we will further discuss the methods used and provide a detailed analysis of the experimental results.  